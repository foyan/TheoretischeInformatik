% !TEX TS-program = pdflatex
% !TEX encoding = UTF-8 Unicode

% This is a simple template for a LaTeX document using the "article" class.
% See "book", "report", "letter" for other types of document.

\documentclass[11pt]{article} % use larger type; default would be 10pt

\usepackage[utf8]{inputenc} % set input encoding (not needed with XeLaTeX)
\usepackage{tikz}
%%% Examples of Article customizations
% These packages are optional, depending whether you want the features they provide.
% See the LaTeX Companion or other references for full information.

%%% PAGE DIMENSIONS
\usepackage{geometry} % to change the page dimensions
\geometry{a4paper} % or letterpaper (US) or a5paper or....
% \geometry{margins=2in} % for example, change the margins to 2 inches all round
% \geometry{landscape} % set up the page for landscape
%   read geometry.pdf for detailed page layout information

\usepackage{graphicx} % support the \includegraphics command and options

% \usepackage[parfill]{parskip} % Activate to begin paragraphs with an empty line rather than an indent

%%% PACKAGES
\usepackage{booktabs} % for much better looking tables
\usepackage{array} % for better arrays (eg matrices) in maths
\usepackage{paralist} % very flexible & customisable lists (eg. enumerate/itemize, etc.)
\usepackage{verbatim} % adds environment for commenting out blocks of text & for better verbatim
\usepackage{subfig} % make it possible to include more than one captioned figure/table in a single float
% These packages are all incorporated in the memoir class to one degree or another...
\usepackage{url}
\usepackage{hyperref}

%%% HEADERS & FOOTERS
\usepackage{fancyhdr} % This should be set AFTER setting up the page geometry
\pagestyle{fancy} % options: empty , plain , fancy
\renewcommand{\headrulewidth}{0pt} % customise the layout...
\lhead{}\chead{}\rhead{}
\lfoot{}\cfoot{\thepage}\rfoot{}

%%% SECTION TITLE APPEARANCE
\usepackage{sectsty}
\allsectionsfont{\sffamily\mdseries\upshape} % (See the fntguide.pdf for font help)
% (This matches ConTeXt defaults)

%%% ToC (table of contents) APPEARANCE
\usepackage[nottoc,notlof,notlot]{tocbibind} % Put the bibliography in the ToC
\usepackage[titles,subfigure]{tocloft} % Alter the style of the Table of Contents
\renewcommand{\cftsecfont}{\rmfamily\mdseries\upshape}
\renewcommand{\cftsecpagefont}{\rmfamily\mdseries\upshape} % No bold!

\renewcommand{\d}{\operatorname{d}}

%%% END Article customizations

%%% The "real" document content comes below...

\usepackage{amsfonts}
\usepackage{amsmath}
\usepackage{amsthm}
\usepackage{amssymb}
\usepackage{wasysym}

\theoremstyle{definition}
\newtheorem*{beispiel}{Beispiel}
\newtheorem{definition}{Definition}
\newtheorem*{bemerkung}{Bemerkung}
\newtheorem*{beweis}{Beweis}
\newtheorem*{ubung}{Übung}

\title{Selbststudium 6}
\author{Florian Lüthi}
%\date{} % Activate to display a given date or no date (if empty),
         % otherwise the current date is printed 

\begin{document}
\maketitle

\section*{Aufgabe 1}

Gelesen.

\section*{Aufgabe 2}

\begin{enumerate}[(a)]

\item Dasdada:
\[
3n \cdot \sqrt{n} \in \mathcal{O}(n^2)
\]

ist äquivalent zu
\[
3n \cdot \sqrt{n} \in \{ r: \mathbb{N} \rightarrow \mathbb{R}^+ | \exists c, n_0 \in \mathbb{N} \textrm{, so dass für alle $n > n_0$: } r(n) \le c \cdot n^2 \}.
\]

Also suchen wir entsprechende $c$ und $n_0$. Wenn wir $c = 3$ wählen und $r(n_0)$ mit $f(n_0)$ gleichsetzen, bekommen wir:
\[
3n_0 \cdot \sqrt{n_0} = 3\cdot {n_0}^2
\]
und damit
\[
n \cdot \sqrt{n_0} = {n_0}^2.
\]
Die Lösung ist $n_0 \in \{0, 1\}$, das heisst ab $n = 1$ würde die Bedingung gelten.

Da offensichtlich ab $n = 1$ $n^2$ schneller wächst als $n\cdot \sqrt{n} = n^1 \cdot n^{0.5} = n^{1.5}$, haben wir $c =3$ und $n_0 = 0$ gefunden und das eingangs Behauptete gezeigt.

\item Nun...
\begin{itemize}

\item ${\mathbf{f_1(n) = 2^{n+3} \in \mathcal{O}(3^n)}}$, denn äquivalent wäre $2^{n+3} = o(3^n)$ und darum
\[
\lim_{n\rightarrow \infty} \frac{2^{n+3}}{3^n} = 0.
\]

Formen wir ein wenig um, bekommen wir
\[
\lim_{n\rightarrow \infty} \frac{2^{n+3}}{2^{n\cdot \log_2(3)}}  = \lim_{n\rightarrow \infty} 2^{n + 3 - n\cdot \log_2(3)} = \lim_{n\rightarrow \infty} 2^{3-0.58n} = 0.
%= \lim_{n\rightarrow \infty} \frac{n + 3}{n\cdot \log_2(3)}
\]

\item ${\mathbf{f_2(n) = n\cdot 2^n \in \mathcal{O}(3^n)}}$, denn äquivalent wäre $n \cdot 2^n = o(3^n)$ und darum
\[
\lim_{n\rightarrow \infty} \frac{n \cdot 2^n}{3^n} = 0.
\]

Formen wir ein wenig um, bekommen wir
\[
\lim_{n\rightarrow \infty} \left[n \cdot \left(\frac{2}{3}\right)^n\right]
\]

Und nach Anwendung der de l'Hospital'schen Regel (wegen $\lim_{x\rightarrow\infty} f(x) = \infty$ und $\lim_{x\rightarrow\infty} g(x) = \infty$ gilt $\lim_{x\rightarrow\infty} [f(x)\cdot g(x)] = \lim_{x\rightarrow\infty} [f(x)'\cdot g(x)']$)
\[
\lim_{n\rightarrow \infty} \left[1 \cdot \left(\frac{2}{3}\right)^n \cdot \ln\frac{2}{3} \right] = 0.
\]

\item ${\mathbf{f_3(n) = 2^{2n} \notin \mathcal{O}(3^n)}}$. Denn angenommen, es gälte tatsächlich ${{2^{2n} \in \mathcal{O}(3^n)}}$, gälte ebenso
\[
\lim_{n\rightarrow \infty} \frac{2^{2n}}{3^n} = 0.
\]
Ein paar Takte Algebra und Infinitesimalrechnung zeigen aber:
\[
\lim_{n\rightarrow \infty} \frac{2^{2n}}{3^n} = \lim_{n\rightarrow \infty} \frac{2^{2n}}{2^{n\cdot \log_2(3)}} = \lim_{n \rightarrow \infty} 2^{2n - n\cdot \log_2(3)} = \lim_{n\rightarrow \infty} 2^{0.42n} = \infty,
\]
was im Widerspruch zur Annahme steht, also muss die Annahme falsch sein.

\end{itemize}

\item Dasdada:
\[
n - \log_2(n) \in \Omega(n)
\]
gilt, weil sich problemlos $n_0 = 4, d = 2$ finden lassen.

\end{enumerate}

\section*{Aufgabe 3 = Aufgabe 6.2}

\begin{enumerate}[(a)]

\item $2^n \in \Theta(2^{n+a})$. Die obere Schranke funktioniert offensichtlich, die untere Schranke genau dann, wenn $d \ge 2^a$ gesetzt wird:
\begin{eqnarray*}
2^{n_0} &=& \frac{1}{d} \cdot 2^{n_0 + a} \\
&=& \frac{1}{d} \cdot 2^{n_0}\cdot 2^a\\
\Rightarrow d\cdot 2^{n_0} &=& 2^{n_0}\cdot 2^a \\
d &=& 2^a
\end{eqnarray*}

\item $2^{b\cdot n} \notin \Theta(2^n)$. Die untere Schranke funktioniert offensichtlich, für die obere Schranke gilt leider im Allgemeinen (für $b = 0$ wäre die Lösung $1 \neq 0$):
\begin{eqnarray*}
\lim_{n \rightarrow \infty} \frac{2^{b\cdot n}}{2^n} = \lim_{n \rightarrow \infty} \frac{\left(2^n\right)^b}{2^n} = \lim_{n \rightarrow \infty} \left(2^n\right)^{b-1} = \infty
\end{eqnarray*}

\item $\log_b n \in \Theta(\log_a n)$\footnote{$c$ umbenannt in $a$, damit es kein Durcheinander mit der Konstanten in der Definition der oberen Schranke gibt.}, weil $\log_a n = \log_b n \cdot \frac 1 {\log_b a}$. Also kann für die obere Schranke $c  = \lceil\frac 1 {\log_b a} \rceil$ und für die untere Schranke $d = \lfloor\frac 1 {\log_b a} \rfloor$ gesetzt werden (gerundet deshalb, weil von der in der Lektüre verwendeten Definition $c, d \in \mathbb{N}$ gefordert wird).

\item $(n+1)! \notin \mathcal{O}(n!)$, denn es gilt
\[
\lim_{n \rightarrow \infty} \frac{(n+1)!}{n!} = \lim_{n \rightarrow \infty} \frac{(n+1)n!}{n!} = \lim_{n \rightarrow \infty} (n+1) = \infty.
\]

\end{enumerate}

\end{document}
