% !TEX TS-program = pdflatex
% !TEX encoding = UTF-8 Unicode

% This is a simple template for a LaTeX document using the "article" class.
% See "book", "report", "letter" for other types of document.

\documentclass[11pt]{article} % use larger type; default would be 10pt

\usepackage[utf8]{inputenc} % set input encoding (not needed with XeLaTeX)
\usepackage{tikz}
%%% Examples of Article customizations
% These packages are optional, depending whether you want the features they provide.
% See the LaTeX Companion or other references for full information.

%%% PAGE DIMENSIONS
\usepackage{geometry} % to change the page dimensions
\geometry{a4paper} % or letterpaper (US) or a5paper or....
% \geometry{margins=2in} % for example, change the margins to 2 inches all round
% \geometry{landscape} % set up the page for landscape
%   read geometry.pdf for detailed page layout information

\usepackage{graphicx} % support the \includegraphics command and options

% \usepackage[parfill]{parskip} % Activate to begin paragraphs with an empty line rather than an indent

%%% PACKAGES
\usepackage{booktabs} % for much better looking tables
\usepackage{array} % for better arrays (eg matrices) in maths
\usepackage{paralist} % very flexible & customisable lists (eg. enumerate/itemize, etc.)
\usepackage{verbatim} % adds environment for commenting out blocks of text & for better verbatim
\usepackage{subfig} % make it possible to include more than one captioned figure/table in a single float
% These packages are all incorporated in the memoir class to one degree or another...
\usepackage{url}
\usepackage{hyperref}

%%% HEADERS & FOOTERS
\usepackage{fancyhdr} % This should be set AFTER setting up the page geometry
\pagestyle{fancy} % options: empty , plain , fancy
\renewcommand{\headrulewidth}{0pt} % customise the layout...
\lhead{}\chead{}\rhead{}
\lfoot{}\cfoot{\thepage}\rfoot{}

%%% SECTION TITLE APPEARANCE
\usepackage{sectsty}
\allsectionsfont{\sffamily\mdseries\upshape} % (See the fntguide.pdf for font help)
% (This matches ConTeXt defaults)

%%% ToC (table of contents) APPEARANCE
\usepackage[nottoc,notlof,notlot]{tocbibind} % Put the bibliography in the ToC
\usepackage[titles,subfigure]{tocloft} % Alter the style of the Table of Contents
\renewcommand{\cftsecfont}{\rmfamily\mdseries\upshape}
\renewcommand{\cftsecpagefont}{\rmfamily\mdseries\upshape} % No bold!

%%% END Article customizations

%%% The "real" document content comes below...

\usepackage{amsfonts}
\usepackage{amsmath}
\usepackage{amsthm}
\usepackage{amssymb}

\theoremstyle{definition}
\newtheorem*{beispiel}{Beispiel}
\newtheorem{definition}{Definition}
\newtheorem*{bemerkung}{Bemerkung}
\newtheorem*{beweis}{Beweis}
\newtheorem*{ubung}{Übung}

\title{Selbststudium 2}
\author{Florian Lüthi}
%\date{} % Activate to display a given date or no date (if empty),
         % otherwise the current date is printed 

\begin{document}
\maketitle

\section*{Aufgabe 2}

\begin{enumerate}[(a)]

\item
\[
\begin{array}{rcll}
&& ((a+b)^* + \varepsilon)^* \\
&=& (\varepsilon + (a+b)^*)^* & \textrm{(wegen $\alpha + \beta = \beta + \alpha$)} \\
&=& ((a+b)^*)^* & \textrm{(wegen $(\varepsilon + \alpha)^* = \alpha^*$)} \\
&=& (a+b)^* & \textrm{(wegen $(\alpha^*)^* = \alpha^*$)}
\end{array}
\]

\item
\[
\begin{array}{rcll}
&& (a\cdot \varnothing ^* \cdot b + (b\cdot a + \varepsilon))^* \\
&=& (a \cdot \varepsilon \cdot b + (b\cdot a + \varepsilon))^* & \textrm{(wegen $\varnothing^* = \varepsilon$)} \\
&=& (a \cdot b + (b\cdot a + \varepsilon))^* & \textrm{(wegen $\alpha \cdot \varepsilon = \alpha$)} \\
&=& ((a\cdot b + b \cdot a ) + \varepsilon)^* & \textrm{(wegen $\alpha + (\beta + \gamma) = (\alpha + \beta) + \gamma$)} \\
&=& ((a\cdot b + b\cdot a))^* & \textrm{(wegen $(\varepsilon + \alpha)^* = \alpha^*$)} \\
&=& (a\cdot b + b\cdot a)^* &  \\

\end{array}
\]

\end{enumerate}

\section*{Aufgabe 3}

\begin{enumerate}[(a)]

\item Gilt $(\alpha\beta + \beta\alpha)^* = (\alpha\beta)^*$? Nein. Seien $\alpha = a, \beta = b$. Dann ist $ba \in (ab + ba)^*$, aber $ba \notin (ab)^*$, also folgt $(ab + ba)^* \not\subseteq (ab)^*$ und dadurch $(ab + ba)^* \neq (ab)^*$ wegen Nichtzutreffen von $A = B \Leftrightarrow A \subseteq B \land B \subseteq A$.

Oder anders ausgedrückt: In $\mathcal{L}((\alpha\beta)^*)$ endet jedes Wort auf $\beta$ (wenn es nicht leer ist), was für $\beta\alpha \in \mathcal{L}((\alpha\beta + \beta\alpha)^*)$ offensichtlich nicht zutrifft.

\item Gilt $(\alpha^*\beta^* + \beta^*\alpha^*)^* = (\alpha^*\beta^*)^*$? Intuitiverweise ja. Versuch über algebraische Umformungen (funktioniert wegen dem zweiten Schritt wohl nicht, aber unter dem Schutz des Kleene'schen Sterns ist ja fast alles erlaubt):
\begin{eqnarray*}
&& (\alpha^*\beta^*)^* \\
&=& (\alpha^*\beta^*)^*(\alpha^*\beta^*)^* \\
&=& (\alpha^*\beta^*\alpha^*\beta^*)^* \\
&=& (\alpha^*\beta^*\alpha^*\beta^* + \alpha^*\beta^*\alpha^*\beta^*)^* \\
&=& (\alpha^*\beta^*\alpha^* + \beta^*\alpha^*\beta^*)^* \\
&=& (\alpha^*\beta^* + \beta^*\alpha^*)^*
\end{eqnarray*}

Der zweite Versuch funktioniert so: Es gilt $ (\alpha^*\beta^*)^* = (\alpha + \beta)^*$, wie Hopcroft und Ullman auf der nächsten nicht mehr kopierten Seite beweisen (weil durch Konkretisierung der RA durch Ersetzen von $\alpha$ und $\beta$ mit $a$ und $b$ klar wird, dass es sich jeweils um dieselbe Sprache $\{a, b\}^*$ handelt). Da aber $ \alpha^*\beta^* \subseteq \alpha^*\beta^* + \alpha^*\beta^*$ gelten muss, und darum auch $ (\alpha^*\beta^*)^* \subseteq (\alpha^*\beta^* + \alpha^*\beta^*)^*$ (weil das Resultat der Vereinigung nicht kleiner als ihre Operanden sein kann), und $(\alpha^*\beta^*)^* = (\alpha + \beta)^*$ schon alles abdeckt, was aus $\alpha$ und $\beta$ überhaupt gebildet werden kann, kann $(\alpha^*\beta^* + \alpha^*\beta^*)^*$ auch nicht grösser sein als $(\alpha^*\beta^*)^*$, ergo muss die Behauptung wahr sein.

\item Gilt $\beta(\alpha\beta)^* = (\beta\alpha)^*\beta$? Ja. Mit Induktion können wir das zeigen.

Verankerung:
\[
\beta(\alpha\beta)^0 = \beta\varepsilon = \beta = \varepsilon\beta = (\beta\alpha)^0\beta
\]

Schritt:
\begin{eqnarray*}
\beta(\alpha\beta)^{i+1} &=& \beta(\alpha\beta)^{i}(\alpha\beta) \\
&=& \beta\alpha(\beta\alpha)^{i}\beta \\
&=& (\beta\alpha)(\beta\alpha)^{i}\beta \\
&=& (\beta\alpha)^{i+1} \beta
\end{eqnarray*}

\end{enumerate}

\end{document}
